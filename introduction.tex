\section{Introduction}

\glspl{MSR} exhibit improved safety characteristics over light-water reactors and other advanced
solid-fueled reactors due to their large fuel expension reactivity coefficients and large
margin-to-boiling from the liquid fuel form of uranium halides dissolved in molten salt coolant
\cite{dolan_1_2017}.
The fuel form also allows for online fuel reprocessing which reduces the amount of excess
reactivity required to keep the core critical during operation. Nevertheless, control rods remain
essential components in \glspl{MSR} for accident prevention and facilitate simpler reactor control
procedures during start-up, shut-down, and load-following operations. It is therefore important to
characterize and accurately model control rod effects through time-dependent multiphysics
simulations to ensure safe reactor operation.

Few existing \gls{MSR} multiphysics studies explicitly include control rods in their models. For
instance, when modeling the \gls{MSRE} pump start-up and coast-down experiments at \gls{ORNL} which
involved control rod movement, most numerical studies simulate the reactivity effects of the
control rods by scaling the neutron source term by the neutron multiplication factor to keep their
reactor model at
criticality \cite{delpech_benchmark_2003, krepel_dyn3d-msr_2007}. Some studies include control rod
models in their steady-state calculations. However, they resort to neutron source term scaling for
the transient calculations due to the inaccuracy of neutron diffusion, $P_1$, and $SP_N$ methods in
highly neutron absorbing regions \cite{kophazi_development_2009, jaradat_development_2021,
yang_development_2022}.

Others adopted homogenization of the control rods and adjacent regions in line with
homogenization-based neutronics methods. Kophazi et al.\ \cite{kophazi_development_2009} modeled
control rods as homogenized hexahedrons in their 3-D Cartesian geometry of the \gls{MSRE}
and imposed albedo boundary conditions for the thermal neutron group. Jaradat et al.\
\cite{jaradat_development_2021} and Yang et al.\ \cite{yang_development_2022} modeled the
\gls{MSRE} control rods as homogenized wedges in their R-$\theta$-Z mesh due to the constraints of
the nodal neutronics solver. Cui et al.\ \cite{cui_development_2021} also homogenized the control
rods as regular hexahedral nodes in accordance with their nodal solver in Cartesian geometry. A
disadvantage of homogenization-based methods is that it removes much of the small-scale
heterogeneity in the flux and temperature. Obtaining a heterogeneous temperature distribution is
especially important in \glspl{MSR} due to the combined fuel-coolant and the positive temperature
reactivity feedback observed in graphite under certain conditions \cite{mathieu_thorium_2006}.

The main difficulty with control rod modeling lies with control rods inducing highly anisotropic
neutron angular fluxes and sharp gradients in the neutron flux in their vicinity. Control rods also
cause shifts in the neutron energy spectrum because their absorption cross sections are much higher
in the lower neutron energy range; more energetic neutrons generally have a higher probability of
escaping the control rod region. As a consequence, modeling control rods accurately requires
high-fidelity neutron transport methods to capture the angular dependence in the neutron flux near
the highly absorbing medium. However, neutron transport methods, such as Monte Carlo, $S_N$, and
$P_N$ methods, are computationally expensive and thus are mainly used for time-independent
neutronic analyses. For time-dependent multiphysics
simulations coupling neutronics to \gls{TH} and other physics present in nuclear reactors, most
reactor analyses rely on neutron diffusion theory.

Neutron diffusion-based solvers may be augmented with transport-derived techniques to improve
solution accuracy. The multilevel \gls{QD} method is one such technique that involves the
generating closure terms, known as Eddington tensors, from high-order neutron transport iterations
that feed into low-order multigroup and one-group \gls{QD} equations
\cite{goldin_quasi-diffusion_1964, anistratov_solution_1986}. This method has been applied to 1-D
time-dependent and 2-D steady-state multiphysics problems \cite{tamang_multilevel_2014,
reynolds_analysis_2023} by iteratively coupling the heat transfer equations to the low-order
one-group \gls{QD} equations, thereby creating an efficient multiphysics coupling scheme. This
method remains computationally intensive for large 3-D problems because it retains the
computational cost of a full-core $S_N$ calculation.

Computational costs can be lowered through domain decomposition schemes.
The multischeme method, implemented in the Griffin reactor physics application
\cite{yang_development_2022}, limits the computational costs of neutron transport calculations
by dividing the problem domain
into subdomains treated using the $S_N$, $P_N$, or neutron diffusion methods. Neutron transport
methods are typically applied near strong neutron absorbers and highly dissimilar material
interfaces where the neutron flux exhibits strong angular dependence. The neutron diffusion method
is applied in the remaining subdomains to reduce computational cost. Adjoining subdomains are
coupled through
Lagrange multiplier interface conditions or the upwinding method applied to neutron surface
currents. Anistratov \& Stehle \cite{anistratov_computational_2012} developed an adaptive domain
decomposition scheme that uses Eddington tensor estimates as a metric for deducing whether a region
is sufficiently diffusive characteristics. It combines high-order $S_N$ equations with low-order
second-moment equations from the Second-Moment method \cite{lewis_comparison_1976} to account for
transport effects in non-diffusive regions. The low-order equations reduce to the neutron diffusion
equations without high-order closures in diffusive regions and naturally enforce flux continuity
across subdomain interfaces.

This paper introduces a new hybrid $S_N$-diffusion neutronics method that also employs domain
decomposition to leverage accurate control rod modeling capabilities of the $S_N$ neutron transport
method while simultaneously retaining the computational efficiency of the neutron diffusion method.
The computationally expensive $S_N$ method problem domain is limited to small subdomains
encompassing the control rods similar to how the multischeme method is used in Griffin. The
hybrid neutronics implementation adopts an iterative two-level structure similar to the method by
Anistratov \& Stehle with a novel adaptive interface coupling between the $S_N$ and neutron
diffusion subdomains. We implemented this hybrid $S_N$-diffusion method in Moltres
\cite{lindsay_moltres_2017, park_verification_2022}, an open-source multiphysics reactor simulation
software developed for \gls{MSR} modeling. The results demonstrate the accuracy of the hybrid
$S_N$-diffusion method on progressively complex 1-D, 2-D, and 3-D neutronics problems modeled after
the \gls{MSRE} reactor which operated at \gls{ORNL} during the 1960s.

TODO: Add outline

%\section{Research Objectives and Outline}
%
%The overarching goal of this work is to improve on Moltres as a reliable, intermediate-fidelity
%simulation tool for multiphysics \gls{MSR} analysis that can be run on workstations as well as on
%large leadership-class computing clusters. Thus, new and existing capabilities in Moltres must
%be accurate within accepted bounds, rigorously verified, computationally efficient, and highly
%scalable. These conditions form the underlying development principles of this work.
%
%The scope of this dissertation can be divided into two main objectives.
%
%    The second objective is to develop, implement, and demonstrate a novel hybrid $S_N$-diffusion
%    method for accurate control rod modeling in
%    time-dependent \gls{MSR} analyses. The hybrid method uses the discrete ordinates ($S_N$)
%    neutron transport method to generate transport corrections in regions near control rods for
%    drift correction terms in modified neutron diffusion equations. The $S_N$ method is applied on
%    a reduced problem domain to ensure the hybrid method remains tractable on small to moderate
%    computing clusters.
%
%Chapter 2 presents a literature review of existing multiphysics \gls{MSR} simulation software,
%\gls{VV} studies on \gls{MSR} modeling and simulation, turbulence modeling in \gls{MSR} systems,
%and transport-correction techniques for neutron diffusion methods.
%Chapter 3 provides an in-depth description of Moltres and its existing capabilities in the context
%of previously published work. This chapter then presents Moltres \gls{VV} results for the CNRS
%benchmark and the numerical \gls{MSRE} zero-power pump experiment studies. Following these studies,
%the chapter presents the implementation and verification of the Spalart-Allmaras turbulence model
%in Moltres.
%Chapter 4 presents the theory and numerical implementation of the hybrid
%$S_N$-diffusion method in Moltres. These include the $S_N$ method implementation, the transport
%correction formulations, the iteration algorithm, the $S_N$-diffusion coupling implementation, and
%general implementation details relating to the underlying numerical solver in Moltres.
%Chapter 5 presents the verification and demonstration of the hybrid $S_N$-diffusion method for
%$k_\text{eff}$ and rod worth calculations through
%$k$-eigenvalue simulations of 1-D, 2-D, and 3-D \gls{MSRE} models. The hybrid method is verified
%against reference results from OpenMC Monte Carlo neutron transport code and \gls{MSRE}
%experimental data.
%Chapter 6 presents a demonstration of the hybrid $S_N$-diffusion method in a time-dependent
%reactivity-initiated simulation modeling a \gls{MSRE} rod drop experiment.
%Chapter 7 concludes this dissertation by summarizing the results presented, identifying limitations
%in this work, and providing potential research directions to address those limitations or extend
%this work.
