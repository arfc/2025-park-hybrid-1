\section{Conclusions} \label{sec:conclusion}

This paper introduced the hybrid $S_N$-diffusion method for \gls{MSR} control rod neutronics
modeling. The hybrid $S_N$-diffusion method improves on the standard neutron diffusion method by
iteratively applying transport corrections generated from solving the $S_N$ neutron transport
method in subdomains containing highly neutron-absorbing control rods. The hybrid method uses the
\gls{SAAF} formulation of the $S_N$ equations, which are highly efficient and
scalable with HYPRE multigrid preconditioners \cite{hypre_hypre_2022} from PETSc.
Drift closure terms computed from the $S_N$ transport solution are passed as transport corrections
in modified forms of the neutron diffusion equations to correct flux errors in near control rods
where diffusion theory is not valid. This work also developed
an adaptive boundary coupling algorithm to couple the $S_N$ and neutron diffusion solvers. The
algorithm automatically truncates transport correction parameters near the boundaries of the
$S_N$ problem subdomain to discard inaccurate correction parameters and preserve smooth neutron
flux gradients across the interface. The $S_N$ and diffusion solvers are coupled through fixed
point iterations using the \gls{MOOSE} \texttt{MultiApp} and \texttt{Transfers} systems.

This paper presented 1-D and 2-D $k$-eigenvalue simulation results with the hybrid method on
models derived from the \gls{MSRE} reactor.
%The 1-D simulations involve six test cases with increasing
%geometric complexity modeled after the air-filled control rod thimbles, salt-graphite lattice
%structure, and vessel regions in the \gls{MSRE}.
Simulations with the OpenMC Monte Carlo neutron
transport code and the $S_8$ method in Moltres provided reference solutions for the hybrid and
neutron diffusion methods to be assessed against. Some discrepancies arose from the eight neutron
energy group structure as evidenced by differences in $k_\text{eff}$ and flux distributions between
OpenMC simulations on continuous energy (OpenMC-CE) and multigroup (OpenMC-MG) modes. Otherwise,
the multigroup $S_8$ method showed good agreement with OpenMC-MG. Although $k_\text{eff}$ estimates
from the hybrid method deviated from the $S_8$ method and OpenMC-MG by 100-400 pcm, the hybrid
method accurately reproduced control rod worth estimates within 0.5 \% of them (2-3 \% relative to
OpenMC-CE). In comparison, the neutron diffusion method fared significantly worse at 5.5 \%
relative to $S_8$ and OpenMC-MG, and 8 \% relative to OpenMC-CE. Our analyses showed little impact
on control rod worth estimates from varying the $S_N$-resolved subdomain size as long as the size
is kept constant between $k$-eigenvalue simulations used to calculate the rod worth. The hybrid
$S_N$-diffusion method exhibited a superlinear convergence rate with the number of fixed point
iterations, leading to the $k_\text{eff}$ error estimate falling below $10^{-7}$ after two
iterations.

The hybrid method maintained its accurate control rod modeling capability in 2-D simulations
involving incremental insertions of three control rods in the \gls{MSRE} model.
The hybrid method reported rod worth error magnitudes of less than 40 pcm relative to OpenMC-CE,
which were significant improvements over the neutron diffusion method error estimates ranging from
569 pcm to 1484 pcm. The hybrid method also showed significant improvements in the absolute mean
and maximum errors in fuel channel power distributions over the neutron diffusion method.

An extension to this work would be to resolve the discrepancies that persist in the hybrid method
$k_\text{eff}$ estimates of individual reactor states. Neutron leakage error at the external
boundaries is likely the most significant source of discrepancy and could be reduced through known
neutron leakage correction techniques such as group-wise or matrix albedo boundary conditions.
Further work is ongoing for the demonstration and performance characterization of the hybrid method
in $k$-eigenvalue and transient 3-D \gls{MSRE} simulations. Transient simulations under active
study include the zero-power rod drop and fractional-power reactivity insertion experiments
performed on the \gls{MSRE}. Future work would include asymmetrical power transients such as
partial channel flow blockage and potential in-core natural circulation flow following a loss of
pump power.

%The 3-D simulations in Chapter \ref{chap:msre} doubled as validation for the 3-D \gls{MSRE} model
%against reference \gls{MSRE} experimental data and the \gls{MSRE} numerical benchmark study
%\cite{fratoni_molten_2020} in
%the International Reactor Physics Experiment Evaluation Project (IRPhEP) handbook. OpenMC and
%hybrid method estimates of $k_\text{eff}$ of the \gls{MSRE} when the experiment first achieved criticality
%$k_\text{eff}$ estimates of the \gls{MSRE} at its initial critcality configuration from OpenMC, the
%hybrid method, and the neutron diffusion method in this work showed good agreement with the Serpent
%model from the \gls{MSRE} numerical benchmark. All numerical estimates exceeded the experimental
%value by 1-2 \% due to possible biases and uncertainties in the nuclear data library for graphite
%\cite{fratoni_molten_2020}. Temperature reactivity coefficient values from this work also showed
%good agreement with \gls{MSRE} data, within experimental uncertainty, with percentage discrepancies
%of about 3 \%. In the subsequent control rod worth study, OpenMC and the hybrid method showed nearly
%perfect overlap in the integral rod worth curve throughout the entire length of rod travel. Due to
%geometric approximations of the control rods in the numerical models, they overestimated the total
%rod worth relative to \gls{MSRE} data by approximately 4-5 \%. The hybrid method significantly
%outperforms the neutron diffusion method which overestimated the total worth by 21.1 \%.
%When comparing solution times, the hybrid method took approximately four times as long as the
%neutron diffusion method.
%The 3-D hybrid method simulations exhibited nearly linear scaling in strong scaling tests, which
%is promising for future projects involving larger reactor models. Some performance optimizations
%may be possible to reduce data transfer times between the $S_N$ and diffusion solvers which took
%up about 18 \% of the total solution time due to inter-processor communications.

%Lastly, Chapter \ref{chap:transient} demonstrated the hybrid method in a time-dependent simulation
%based on a \gls{MSRE} zero-power rod drop experiment \cite{prince_zero-power_1968}.
%The simulation required coupling the hybrid method solver to the pre-existing \gls{DNP} solver
%in Moltres through a nested iteration coupling setup.
%I set up the rod drop simulation through a
%$k$-eigenvalue simulation with the control rod at its initial height before using the simulation
%output as the initial condition for the time-dependent rod drop simulation. Moltres reproduced the
%expected prompt and delayed response in the integral neutron count data observed in \gls{MSRE}
%experimental data. Convergence issues midway through the simulation had a minor impact on the
%solution precision. Overall, this work successfully demonstrated a time-dependent
%reactivity-initiated simulation using the hybrid method.

%\subsection{Limitations and Future Work}

%The second \gls{VV} study verified the looped \gls{DNP} flow modeling capability in Moltres under
%steady-state and transient scenarios. In the original pump experiments, the control rod was driven
%by a ``flux servo controller'' that adjusts the rod position in response to flux changes to maintain
%criticality. Our study in this work approximated this action through $k$-eigenvalue solvers
%coupled to time-dependent \gls{DNP} flow solvers. Potential future work would be to implement a
%control system such as a PID (Proportional-Integral-Derivative) controller in combination with
%the newly-implemented hybrid method to create a more representative model of the pump experiments.
%This would enhance model validation and open up additional research directions which require a
%control system.
%
%This work implemented and verified the Spalart-Allmaras turbulence model in Moltres. The turbulence
%model \gls{VV} tests indicated significant mesh refinement requirements near the wall. Mesh
%refinement scales with the flow Reynolds number. With \gls{MSR} designs such as the \gls{MSFR}
%reaching Reynolds numbers on the order of $10^6$, future work in this area should focus on
%implementing wall functions. Wall functions eliminate the need for fine mesh near the wall by
%approximating the log-law velocity profile near the wall. Verification and demonstration of the
%turbulence model, beyond general \gls{VV} tests, in \gls{MSR} turbulent salt flow problems is also
%crucial for the continued development of Moltres as a \gls{MSR} simulation tool.

%The 3-D simulations in this work uncovered significant memory usage by the $S_N$ solver, even
%with the distributed mesh feature to spread mesh and variable data storage across the compute nodes.
%Each full-core simulation required at least 40 nodes with 512 GB memory each to run.
%This issue may be a stumbling block for using the hybrid method on smaller computing clusters that
%have less memory per processor. A custom preconditioner and solver routine in Moltres could help to
%reduce memory usage by referencing the same stored Jacobian C++ variable for neutron angular flux
%variables in the same neutron energy group and on the same \gls{FEM} quadrature point. Their
%Jacobian formulations are identical aside from the level-symmetric ordinate and weight. Another
%area for optimization in the hybrid method is reducing data transfer times between the $S_N$ and
%diffusion solvers. Due to mesh distribution, each 3-D simulation spent 18 \% of its solution time
%on data transfers between processors. This could be minimized through optimizations in the mesh
%distribution and the data transfer caching systems.
%
%Time-dependent simulations in this work uncovered two issues: rod cusping effects when a moving
%control rod does not align with the mesh interfaces, and slow solution convergence rates during
%control rod motion. While this work applied an empirical technique to correct for rod cusping
%effects, Moltres would benefit from a more robust technique for the long term. Slow solution
%convergence rates could be resolved by investigating better nested solver coupling structures to
%mitigate the effects of lagged solutions and applying solution relaxation schemes.
%
%Finally, future work could demonstrate the hybrid method to its fullest potential through the
%simulation of asymmetric reactivity-initiated transients. Asymmetry refers to significant changes
%in the neutron flux shape during a transient scenario. The core benefit of the hybrid method is
%the spatial resolution it provides to time-dependent simulations involving control rod movement.
%This objective could be met through modeling most other \gls{MSR} designs whose control and shim
%rods are not centrally located.
